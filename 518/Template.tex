\documentclass[11pt, ngerman, fleqn, DIV=15, headinclude, BCOR=2cm]{scrreprt}

\usepackage{../../header}

\usepackage{placeins}
\usepackage[maxfloats=50]{morefloats}

\usepackage{csquotes}

\usepackage{tikz}
\usetikzlibrary{chains}
\usetikzlibrary{shapes.geometric}

\tikzset{device/.style={
                rectangle,
                minimum size=6mm,
                draw=black
            },
            monitor/.style={
                rectangle,
                rounded corners=2mm,
                minimum size=6mm,
                draw=black
            },
        }

\usepackage{pgfplots}
\pgfplotsset{
    compat=1.9,
    width=0.8\linewidth,
    xticklabel style={/pgf/number format/use comma},
    yticklabel style={/pgf/number format/use comma},
}
\usepgfplotslibrary{polar}

\usepgfplotslibrary{external}
\tikzexternalize[mode=list and make]
\tikzsetexternalprefix{Abbildung-}

\DeclareSIUnit{\skt}{SKT}

\usepackage{booktabs}

\hypersetup{
    pdftitle=
}

\subject{Praktikumsprotokoll}
\title{Höhenstrahlung}
\subtitle{Versuch P518 -- Universität Bonn}
\author{
    Martin Ueding \\ \small{\href{mailto:mu@martin-ueding.de}{mu@martin-ueding.de}}
    \and
    Lino Lemmer \\
    \small{\href{mailto:l2@uni-bonn.de}{l2@uni-bonn.de}}
}

\date{\daterange{2014-07-02}{2014-07-03}}

\publishers{Tutor: Michael Lupberger}

\begin{document}

\maketitle

\begin{abstract}
    % TODO Abstract formulieren.
\end{abstract}

\tableofcontents

\chapter{Theorie}

\begin{figure}[htbp]
    \centering
    \tikzsetnextfilename{cos2}
    \begin{tikzpicture}
        \begin{polaraxis}[
                xmin=180,
                xmax=360,
            ]
            \addplot[black] table {cos2.csv};
        \end{polaraxis}
    \end{tikzpicture}
    \caption{%
        Erwartete Intensitätsverteilung der Höhenstrahlung $\cos(\phi)^2$.
        Dabei ist $\phi$ hier so gewählt, dass $\phi = \SI{90}{\degree}$
        senkrecht nach oben zeigt. Somit gilt $\theta = \phi + \piup/2$.
    }
    \label{fig:cos2}
\end{figure}

\section{Vorbereitungsfragen zur Winkelverteilung}

\parencite[11]{physik512-Anleitung}

\begin{quote}
    Wie funktionieren Szintillatoren und Photomultiplier? Welche physikalischen
    Prozesse und apparativen Einflüsse bestimmen den zeitlichen Verlauf des
    Photomultiplier-Ausgangssignals? Wie hängt die Pulshöhe von der gewählten
    Hochspannung am Photomultiplier ab?
\end{quote}

% TODO Antwort einfügen.

\begin{quote}
    Wie funktionieren Diskriminator- und Koinzidenz-Schaltungen? Wie sehen die
    jeweiligen Ausgangssignale aus? Wie beeinflusst die Schwellenhöhe eines
    Diskriminators die zeitliche Lage des Ausgangssignals gegenüber dem
    Eingangssignal?
\end{quote}

% TODO Antwort einfügen.

\begin{quote}
    Die Zählrate für jeden einzelnen Zähler betrage etwa 1 Hz am
    Diskriminatorausgang. Wie groß ist die theoretisch zu erwartende Rate der
    Zufallskoinzidenzen für eine der Dreifach- Koinzidenz-Schaltungen zur
    Messung der Winkelverteilung der Höhenstrahlung? Und für die
    Koinzidenz-Schaltung, die zur Messung des Pulshöhenspektrums verwendet
    wird? Wie groß sind jeweils die Totzeiten? Machen Sie einen Vorschlag, wie
    die Rate der Zufallskoinzidenzen für eine der
    Dreifach-Koinzidenz-Schaltungen gemessen werden kann.
\end{quote}

% TODO Antwort einfügen.

\begin{quote}
    Welchen inhaltlichen Zusammenhang haben Bethe-Bloch-Gleichung und
    Landau-Verteilung?
\end{quote}

% TODO Antwort einfügen.

\begin{quote}
    Welche Beiträge hat das Pulshöhenspektrum? Welche physikalischen und
    apparativen Einflüsse bestimmen die Form des Pulshöhenspektrums? Wie
    unterscheiden sich zwei Pulshöhenspektren, wenn diese mit und ohne aktiven
    Gate-Eingang aufgenommen werden?
\end{quote}

% TODO Antwort einfügen.

\begin{quote}
    Wie hängen Pulshöhenspektrum und Schwellenkurve zusammen? Wie ändert sich
    die Form der Schwellenkurve, wenn man die Anzahl der Koinzidenzsignale
    anstelle der Diskriminatorsignale betrachtet?
\end{quote}

% TODO Antwort einfügen.

\begin{quote}
    Welche Funktion erfüllt das in Abbildung 5 gezeigte LabVIEW-Programm?
\end{quote}

% TODO Antwort einfügen.

\section{Vorbereitungsfragen zur Myonenlebensdauer}

\parencite[14]{physik512-Anleitung}

\begin{quote}
    Es kommen sowohl Myonen als auch Antimyonen auf der Erdoberfläche an.
    Welcher atom- physikalische Prozeß ist für Myonen möglich, aber nicht für
    Antimyonen? Wie beeinflußt das qualitativ die in diesem Versuch gemessene
    Lebensdauer?
\end{quote}

% TODO Antwort einfügen.

\begin{quote}
    Machen Sie einen Vorschlag, wie man unter Verwendung von Diskriminatoren,
    Verzögerungs- kabeln und Koinzidenz-Schaltungen die START-und STOP-Impulse
    erzeugen kann.
\end{quote}

% TODO Antwort einfügen.

\begin{quote}
    Entwerfen Sie das Blockschaltbild für Messkreis und Monitorkreis.
\end{quote}

% TODO Antwort einfügen.

\begin{quote}
    Können Sie eine andere Gestaltung der Zeitintervalle vorschlagen, die die
    Genauigkeit der Myonlebensdauerbestimmung optimieren würde?
\end{quote}

% TODO Antwort einfügen.

\begin{quote}
    Das aus der Apparatur kommende Start-Signal wird gegenüber dem Stop-Signal
    um ca. 100 ns verzögert. Warum ist das notwendig? Wie wirkt es sich auf das
    Messergebnis aus?
\end{quote}

% TODO Antwort einfügen.

\begin{quote}
    Wie können scheinbare Myonzerfälle zustandekommen? Welche Messgrößen
    braucht man, um die erwartete Anzahl von Zufallsereignissen berechnen zu
    können?
\end{quote}

% TODO Antwort einfügen.

\chapter{Durchführung Winkelverteilung}

Plan:

\begin{enumerate}
    \item
        Einstellung der Diskriminatorschwelle für D12 um den Untergrund zu
        filtern. Siehe §\ref{sec:einstellung_diskriminatorschwelle}.
        
    \item
        Finden der optimalen Verzögerung der Diskriminatorsignale, damit deren
        Überschneidung maximiert wird.
\end{enumerate}

\section{Aufbau, Geräte}

\subsection{Zählerring}

\subsection{Elektronik}

\subsubsection{Diskriminatoren}

\subsubsection{Verteiler}

\subsubsection{Koinzidenz}

\subsubsection{Zähler}

\section{Justierung}

\subsection{Einstellung der Diskriminatorschwelle}
\label{sec:einstellung_diskriminatorschwelle}

\chapter{Auswertung Winkelverteilung}

\chapter{Durchführung Myonenlebensdauer}

\chapter{Auswertung Myonenlebensdauer}

\chapter{Ergebnis}

\end{document}

% vim: spell spelllang=de tw=79
