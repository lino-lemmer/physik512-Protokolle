\documentclass[11pt, ngerman, fleqn, DIV=15, headinclude, BCOR=2cm]{scrreprt}

\usepackage{../../header}

\usepackage{placeins}
\usepackage[maxfloats=50]{morefloats}

\usepackage{csquotes}

\usepackage{tikz}
\usetikzlibrary{chains}
\usetikzlibrary{shapes.geometric}

\tikzset{device/.style={
                rectangle,
                minimum size=6mm,
                draw=black
            },
            monitor/.style={
                rectangle,
                rounded corners=2mm,
                minimum size=6mm,
                draw=black
            },
        }

\usepackage{pgfplots}
\pgfplotsset{
    compat=1.9,
    width=0.8\linewidth,
    xticklabel style={/pgf/number format/use comma},
    yticklabel style={/pgf/number format/use comma},
}
\usepgfplotslibrary{polar}

\usepgfplotslibrary{external}
\tikzexternalize[mode=list and make]
\tikzsetexternalprefix{Abbildung-}

\DeclareSIUnit{\skt}{SKT}

\usepackage{booktabs}

\hypersetup{
    pdftitle=
}

\subject{Praktikumsprotokoll}
\title{Höhenstrahlung}
\subtitle{Versuch P518 -- Universität Bonn}
\author{
    Martin Ueding \\ \small{\href{mailto:mu@martin-ueding.de}{mu@martin-ueding.de}}
    \and
    Lino Lemmer \\
    \small{\href{mailto:l2@uni-bonn.de}{l2@uni-bonn.de}}
}

\date{\daterange{2014-07-02}{2014-07-03}}

\publishers{Tutor: Michael Lupberger}

\begin{document}

\maketitle

\begin{abstract}
    % TODO Abstract formulieren.
\end{abstract}

\tableofcontents

\chapter{Theorie}

\begin{figure}[htbp]
    \centering
    \tikzsetnextfilename{cos2}
    \begin{tikzpicture}
        \begin{polaraxis}[
                xmin=180,
                xmax=360,
            ]
            \addplot[black] table {cos2.csv};
        \end{polaraxis}
    \end{tikzpicture}
    \caption{%
        Erwartete Intensitätsverteilung der Höhenstrahlung $\cos(\phi)^2$.
        Dabei ist $\phi$ hier so gewählt, dass $\phi = \SI{90}{\degree}$
        senkrecht nach oben zeigt. Somit gilt $\theta = \phi + \piup/2$.
    }
    \label{fig:cos2}
\end{figure}

\chapter{Durchführung Winkelverteilung}

\chapter{Durchführung Myonenlebensdauer}

\chapter{Auswertung Winkelverteilung}

\chapter{Auswertung Myonenlebensdauer}

\chapter{Ergebnis}

\end{document}

% vim: spell spelllang=de tw=79
